\documentclass[10pt]{beamer}

\usetheme[progressbar=frametitle]{metropolis}
\usepackage{appendixnumberbeamer}

\usepackage{booktabs}
\usepackage[scale=2]{ccicons}

\usepackage{pgfplots}
\usepgfplotslibrary{dateplot}

\usepackage{xspace}
\newcommand{\themename}{\textbf{\textsc{metropolis}}\xspace}

\title{ToDoFamily}
\subtitle{Modern, privlačen in intuitiven ToDo list za celotno družino}
\date{Zimski semester, 2018/19}
\author{Aljaž Glavač, Luka Perovič, Andraž Razpor}
\institute{Univerza v Ljubljani, Fakulteta za računalništvo in informatiko}
% \titlegraphic{\hfill\includegraphics[height=1.5cm]{logo.pdf}}

\begin{document}

\maketitle

%\begin{frame}{Table of contents}
%  \setbeamertemplate{section in toc}[sections numbered]
%  \tableofcontents[hideallsubsections]
%\end{frame}

\section{Features}

\begin{frame}{Feature -- list}
    \begin{itemize}
        \item Avtentikacija
        \item Sestavljanje družine -- dodajanje otrok
        \item Vpisovanje opravil
        \item Določanje pomembnosti 
        \item Rok opravljenosti
        \item Dodeljevanje opravila otrokom
        \item Kreiranje urnika
    \end{itemize}
\end{frame}

\begin{frame}{Feature -- roadmap}
\begin{columns}[c]
    \begin{column}{0.1\textwidth}
        \hfill 3. 11 \\
        \hfill 17. 11 \\
        \hfill 26. 11 \\
        \hfill 10. 12 \\
        \hfill 30. 12 \\
        \hfill 15. 01 \\
        \hfill 28. 01 \\
    \end{column}
    \hspace{-15pt}\vrule\hspace{5pt}%
    \begin{column}{0.9\textwidth}  %%<--- here
            Avtentikacija \\
            Sestavljanje družine -- dodajanje otrok \\
            Vpisovanje opravil \\
            Določanje pomembnosti \\
            Rok opravljenosti \\
            Dodeljevanje opravila otrokom \\
            Kreiranje urnika \\
    \end{column}
\end{columns}
\end{frame}

\end{document}

% Commands to include a figure:
%\begin{figure}
%\includegraphics[width=\textwidth]{your-figure's-file-name}
%\caption{\label{fig:your-figure}Caption goes here.}
%\end{figure}

% Commands to include a table
%\begin{table}
%\centering
%\begin{tabular}{l|r}
%Item & Quantity \\\hline
%Widgets & 42 \\
%Gadgets & 13
%\end{tabular}
%\caption{\label{tab:widgets}An example table.}
%\end{table}

